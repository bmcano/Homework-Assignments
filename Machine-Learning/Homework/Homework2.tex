\documentclass[12pt]{article}

% Packages for math symbols and formatting
\usepackage{amsmath}
\usepackage{amssymb}
\usepackage{amsfonts}

% Margins
\usepackage[a4paper, margin=1in]{geometry}

% Title and Author
\title{ECE:5450 - Homework 2}
\author{Brandon Cano}
\date{\today}

\begin{document}

% Title page
\maketitle

\section*{Problem 1}

We know that
\[
r = \begin{bmatrix}
x \\
y
\end{bmatrix},\,
\mu = E[r] = \begin{bmatrix}
1 \\
2
\end{bmatrix},\,
\sum = E\left[(r - \mu)(r - \mu)^{T}\right]= \begin{bmatrix}
1 & 0 \\
0 & 8
\end{bmatrix}
\]
and that $\sum$ is diagonal because of the placement of 0, this means that $x$ and $y$ are independent.
This means that we can split the value $E[x^{2}y]$ into
\[ E[x^{2}y] = E[x^{2}] * E[y] \]
and solve them one at a time.
For $E[x^{2}]$, we have $x$ distributed as $P(x) = \cal{N}$:
\[ 
E[x^{2}] = Var(x) + E[x]^{2}
\]
We know that $E[x] = 1$ because we derive that from $\mu$ and that the $Var(x) = 1$ because the covariance matrix is diagonal and thus independent we can pull the 1 directly from it. 
So we get
\[ 
E[x^{2}] = Var(x) + E[x]^{2} = 1 + 1^{2} = 2
\]
For $E[y]$, we have $y$ distributed as $P(y) = \cal{N}$ and we can see that 
\[
E[y] = 2
\]
Thus,
\[
E[x^{2}y] = E[x^{2}] * E[y] = 2 * 2 = \boxed{4}
\]

\section*{Problem 2}

\begin{enumerate}
    \item[\textbf{a.}] Since $p(x)$ is exponential we just need to maximize $-\frac{1}{2}(x - \mu)^{T}\sum^{-1}(x - \mu)$. 
    To do this we need $(x - \mu) = 0$.
    So this exponential then becomes:
    \[
    -\frac{1}{2}(x - \mu)^{T}\sum^{-1}(x - \mu) = 0
    \]
    \[
    (x - \mu)^{T}\sum^{-1}(x - \mu) = 0
    \]
    \[
    (x - \mu)^{T}(x - \mu) = 0
    \]
    \[
    (x - \mu) = 0
    \]
    Thus we get $exp(0) = 1$ so, 
    \[
    x = \mu
    \]
    \item[\textbf{b.}] Since we know that $x \sim \cal{N}(\mu, \sum)$ we can use the second-moment formula which gives us
    \[
    E[xx^{T}] = Cov(x) + E[x]E[x]^{T}
    \]
    The $Cov(x)$ is given by $x \sim \cal{N}(\mu, \sum)$ so $Cov(x) = \sum$. 
    As well as we know that the expectation of $x$ is $\mu$.
    Thus we get
    \[
    E[xx^{T}] = \sum + \mu\mu^{T}
    \]
	\item[\textbf{c.}] To solve this we need to separate the parts into 2 cases.
	For the case $m = n$, the expectation is the same as in part \textbf{b.} so that means we get
	\[
	E[x_{m}x_{n}^{T}] = \sum + \mu\mu^{T}
	\]
	For the case $m \neq n$, since they are independent the covariance is 0, which means we get
	\[
	E[x_{m}x_{n}^{T}] = \mu\mu^{T}
	\]
	For combining the two parts, we bring in the delta function, since we know in the case of $m = n$ the value is 1 and 0 otherwise, we can add it to the $\sum$ value.
	So by combining these parts we get
	\[
	\boxed{E[x_{m}x_{n}^{T}] = \sum\delta(m - n) + \mu\mu^{T}}
	\]
	
\end{enumerate}

\section*{Problem 3}

We know that $\cal{L}$ $(x, y, \lambda) = f(x, y) + \lambda g(x, y)$ and from the problem we know $f(x, y) = x^{2} + 2y^{2}$ and $g(x, y) = y - 3x - 2$.
Then we start with the min-max problem but we will flip it to start by solving for the min values first so we get
\[
\underset{\lambda}{max}\, \underset{x, y}{min} \, \cal{L} (\textit{x}, \textit{y}, \lambda)
\]
and now we first start solving
\[
\underset{x, y}{min} \,\, x^{2} + 2y^{2} + \lambda (y - 3x - 2)
\]
First we take the partial derivatives with respect to $x$ and $y$.
\[
D_{x, y} \cal{L} (\textit{x}, \textit{y}) = 
\begin{pmatrix}
\frac{\partial\cal{L}}{\partial x} \\
\frac{\partial\cal{L}}{\partial y}
\end{pmatrix} =
\begin{pmatrix}
2x - 3\lambda \\
4y + \lambda
\end{pmatrix} 
\]
Then we solve for $x$ and $y$
\begin{minipage}[t]{0.2\textwidth}
\[
2x - 3\lambda = 0 \\
\]
\[
2x = 3\lambda 
\]
\[
x = \frac{3}{2}\lambda
\]    
\end{minipage}
\hfill
\begin{minipage}[t]{0.2\textwidth}
\[
4y + \lambda = 0 \\
\]
\[
4y = -\lambda
\]
\[
y = \frac{-\lambda}{4}
\]
\end{minipage}

Then we plug these back into the original equation, take its' derivative with respect to $\lambda$ then solve for $\lambda$.
\[
{\cal{L}} (\textit{x}, \textit{y}, \lambda) = 
(\frac{3}{2}\lambda)^{2} + 2(\frac{-\lambda}{4})^{2} + 
\lambda ((\frac{-\lambda}{4}) - 3(\frac{3}{2}\lambda) - 2)
= \frac{-19}{8}\lambda^{2} - 2\lambda
\]
\[
\frac{\partial\cal{L}}{\partial\lambda} = \frac{-19}{4}\lambda - 2 = 0
\]
\[
\lambda = \frac{-8}{19}
\]
Next we take this value and plug it back into the $x$ and $y$ we found earlier
\[
x = \frac{3}{2}*\frac{-8}{19} = \frac{-12}{19}
\]
\[
y = \frac{-1}{4} * \frac{-8}{19} = \frac{2}{19}
\]
and we get
\[
\boxed{(x, y) = \left(\frac{-12}{19}, \frac{2}{19}\right)}
\]


\section*{Problem 4}

We know that $R(h, s) = 4hs$ and the constraint is $20h + 10s = 100$ so we get $\cal{L}$ $(x, y, \lambda) = 4hs + \lambda (100 - 20h - 10s)$.
Then we start with the min-max problem but we will flip it to start by solving for the min values first so we get
\[
\underset{\lambda}{max}\, \underset{x, y}{min} \, \cal{L} (\textit{h}, \textit{s}, \lambda)
\]
and now we first start solving
\[
\underset{x, y}{min} \,\, 4hs + \lambda (100 - 20h - 10s)
\]
First we take the partial derivatives with respect to $h$ and $s$.
\[
D_{h, s} \cal{L} (\textit{h}, \textit{s}) = 
\begin{pmatrix}
\frac{\partial\cal{L}}{\partial h} \\
\frac{\partial\cal{L}}{\partial s}
\end{pmatrix} =
\begin{pmatrix}
4s - 20\lambda \\
4h - 10\lambda
\end{pmatrix} 
\]
Then we solve for $h$ and $s$
\begin{minipage}[t]{0.2\textwidth}
\[
4s - 20\lambda = 0 \\
\]
\[
4s = 20\lambda 
\]
\[
s = 5\lambda
\]    
\end{minipage}
\hfill
\begin{minipage}[t]{0.2\textwidth}
\[
4h - 10\lambda = 0 \\
\]
\[
4h = 10\lambda
\]
\[
h = \frac{5\lambda}{2}
\]
\end{minipage}

Then we plug these back into the original equation, take its' derivative with respect to $\lambda$ then solve for $\lambda$.
\[
{\cal{L}} (\textit{h}, \textit{s}, \lambda) = 
4(\frac{5}{2}\lambda )(5\lambda ) + \lambda (100 - 20(\frac{5}{2}\lambda ) - 10(5\lambda )) = 100\lambda - 50\lambda^{2} = 0
\]
\[
\frac{\partial\cal{L}}{\partial\lambda} = 100 - 100\lambda = 0
\]
\[
\lambda = 1
\]
Next we take this value and plug it back into the $x$ and $y$ we found earlier
\[
h = \frac{5}{2}(1) = \frac{5}{2}
\]
\[
s = 5(1) = 5
\]
and we get
\[
\boxed{R(\frac{5}{2}, 5) = 4 * \frac{5}{2} * 5 = 50}
\]
and we can see that
\[
20 * \frac{5}{2} + 10(5) = \frac{100}{2} + 50 = 100
\]


\end{document}
